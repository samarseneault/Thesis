% Résumé du mémoire.
%
\chapter*{RÉSUMÉ}\thispagestyle{headings}
\addcontentsline{toc}{compteur}{RÉSUMÉ}

Le risque est omniprésent dans nos vies; il est impossible de l'ignorer. Nous pouvons soit l'éviter ou l'affronter. Selon la situation, les deux approches peuvent avoir leur mérite. D'une part, en évitant toujours le danger, nous accomplirons forcément bien peu. D'autre part, en le bravant continuellement, nous courrons à notre perte. Nous choisissons donc le meilleur plan d'action selon la gravité de la situation et selon notre tolérance au risque. Or, le risque n'est pas uniquement présent dans notre quotidien, mais aussi dans des domaines d'applications scientifiques. Plus particulièrement, on s'intéresse ici à sa présence dans le domaine de la robotique en essaim, où des collectivités de robots doivent accomplir des missions dans des environnements souvent périlleux. Sachant ce contexte, comment peut-on inculquer à ces robots une politique de gestion de risque équilibrée afin de les rendre plus résilients?

La stratégie adoptée dans ce mémoire pour répondre à la question précédente est de créer une politique de stockage distribué tenant compte du risque ainsi qu'une manière collaborative d'explorer un environnement inconnu sans trop exposer les individus d'un essaim au danger. Ce faisant, on vérifie la pertinence de la gestion de risque dans deux contextes d'intelligence en essaim différents mais reliés. Le premier système développé dans ce mémoire, \textit{\ac{RASS}}, se base sur le potentiel des robots à stocker de l'information sécuritairement ainsi que sur un mécanisme de percolation des données vers un point central. La seconde idée exposée est \textit{\ac{DORA}}, qui optimise des gradients locaux d'information et de risque pour explorer son environnement optimalement.

Afin d'évaluer chaque système conçu, deux phases sont nécessaires. D'abord, des expériences sont réalisées en simulation pour vérifier conceptuellement les systèmes ainsi que leur capacité d'être mis à l'échelle. Puis, des tests sont effectués en environnement physique (sur de vrais robots) pour s'assurer de leur applicabilité en scénarios réels. Pour \ac{RASS} ainsi que pour \ac{DORA}, les résultats montrent que tout comme avec les humains, une prise en considération modérée du risque selon la situation engendre des résultats optimaux.

\ac{RASS} et \ac{DORA} proposent des pistes pour rendre des applications de robotique en essaim plus résilientes. Conséquemment, ils pourront être adaptés et bonifiés pour des scénarios plus diversifiés et complexes que ceux présentés dans les expériences réalisées ici, tel que des scénarios de recherche et de sauvetage élaborés.

% Le résumé est un bref exposé du sujet traité, des objectifs visés,
% des hypothèses émises, des méthodes expérimentales utilisées et de
% l'analyse des résultats obtenus. On y présente également les
% principales conclusions de la recherche ainsi que ses applications
% éventuelles. En général, un résumé ne dépasse pas quatre pages.

% Le résumé doit donner une idée exacte du contenu du mémoire ou de la thèse. Ce ne
% peut pas être une simple énumération des parties du document, car il
% doit faire ressortir l'originalité de la recherche, son aspect
% créatif et sa contribution au développement de la technologie ou à
% l'avancement des connaissances en génie et en sciences appliquées.
% Un résumé ne doit jamais comporter de références ou de figures.
