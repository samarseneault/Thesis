% Résumé du mémoire.
%
\chapter*{RÉSUMÉ}\thispagestyle{headings}
\addcontentsline{toc}{compteur}{RÉSUMÉ}

Le risque est omniprésent dans nos vies; il est impossible de l'ignorer. Nous pouvons soit l'éviter ou l'affronter. Selon la situation, les deux approchent peuvent avoir leur mérite. D'une part, en évitant toujours le danger, nous accomplirons forcément bien peu. D'autre part, en le bravant continuellement, nous courrons à notre perte. Nous choisissons donc le meilleur plan d'action selon la gravité de la situation et selon notre tolérance au risque. Or, celui-ci n'est pas uniquement présent dans notre quotidien, mais aussi dans des domaines d'applications scientifiques. Plus particulièrement, nous nous intéressons ici à sa présence dans le domaine de la robotique en essaim, où des collectivités de robots doivent accomplir des missions dans des environnements souvent périlleux. Mais comment pouvons nous inculquer à ces robots une politique de gestion de risque équilibrée?

La stratégie adoptée dans ce mémoire pour répondre à la question précédente est de créer une politique de stockage distribué tenant compte du risque ainsi qu'une manière collaborative d'explorer un environnement inconnu sans trop exposer les individus d'un essaim au danger. Ce faisant, on vérifie la pertinence de la gestion de risque dans deux contextes différents mais reliés. Afin de l'évaluer, des expériences tant en simulation qu'en environnement physique (sur de vrais robots) sont réalisés pour chaque système conçu. Dans les deux cas, le résultat est que tout comme avec les humains, une prise en considération modérée du risque selon la situation engendre des résultats optimaux.

Le résumé est un bref exposé du sujet traité, des objectifs visés,
des hypothèses émises, des méthodes expérimentales utilisées et de
l'analyse des résultats obtenus. On y présente également les
principales conclusions de la recherche ainsi que ses applications
éventuelles. En général, un résumé ne dépasse pas quatre pages.

Le résumé doit donner une idée exacte du contenu du mémoire ou de la thèse. Ce ne
peut pas être une simple énumération des parties du document, car il
doit faire ressortir l'originalité de la recherche, son aspect
créatif et sa contribution au développement de la technologie ou à
l'avancement des connaissances en génie et en sciences appliquées.
Un résumé ne doit jamais comporter de références ou de figures.
