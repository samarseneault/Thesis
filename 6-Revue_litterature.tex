\Chapter{LITERATURE REVIEW}\label{sec:RevLitt}
This chapter explores relevant state of the art literature related to the domain of this thesis. This includes insights into the following topics: swarm intelligence, decentralized storage, communication and consensus, distributed storage, risk assessment in robot swarms, and path planning and exploration strategies. Not all articles explored in this review were directly used, but they certainly all provide good background information. The direct contribution of those which were used for either \ac{RASS} or \ac{DORA} is explained in detail in their respective chapters.

\section{Decentralized Storage}
To support collaboration within a swarm, efficient communication is necessary \cite{dutta2020efficient}. To that end, relevant articles supporting distributed or decentralized ways of sharing data are presented here.

One of the first accounts of decentralized data transmission in the context of swarm intelligence was proposed in Ant Colony Optimization \cite{dorigo2006ant}, in which agents leave pheromone trails in their environment which can be interpreted by their peers. This mechanism is a form of stigmergy.

Using the same concept to store data, the virtual stigmergy \cite{pinciroliTuple2016} is a distributed key-value storage mechanism that allows a swarm of robots to efficiently share information at runtime, and effectively constitutes a shared memory space. It was designed to work with robots with limited bandwidth, lossy communication mediums, and low processing capabilities. Another advantage of virtual stigmergy is its inherent integration with Buzz. SwarmMesh \cite{majcherczykSwarmmesh2020}, another distributed key-value storage mechanism, introduces the idea of storing data on the less vulnerable members of the swarm to increase robustness. This is done by calculating keys by hashing the data needing to be stored and then partitioning the keys based on the fitness (connectivity, available memory, etc.) of a given robot to store information. Access to stored information is done through a custom routing algorithm based on the keys. The system also allows for data replication to further improve robustness. Its usefulness has been proven in situations where robot resources and capabilities are limited \cite{majcherczyk2021distributed}. Both virtual stigmergy and SwarmMesh are scalable to large swarms of robots with dynamic connection topologies. However, their main limitation reside mainly in the small size of the data that can be stored in the swarm. This is an issue if larger quantities of data need to be shared. For example, storing software binaries with these systems might not be feasible. SOUL \cite{varadharajan2020soul} addresses this by adapting existing peer-to-peer file sharing mechanisms (which are already decentralized in nature and support large file sharing) to the context of robotic swarms. The key principle in SOUL is that files are "split into a series of smaller chunks of data, referred to as datagrams, spread throughout the swarm". This spreading is done through a bidding process among the members of the swarm to determine which storage configuration will minimize the cost of reconstructing data (i.e. retrieving it) from the distributed datagrams. It might be necessary to use distributed storage mechanism which are optimized for certain metrics. In \cite{amigoni2017multirobot}, a method for scenarios in which bandwidth is limited is suggested. In situations where the location of the storage nodes is important, geographic hash tables can be used \cite{wu2008ldht,ratnasamy2002ght,ahullo2008supporting}. For other constraints such as requirements for low energy usage, Cell Hash Routing  \cite{araujo2005chr} may be considered as a starting point. It is well suited to networks with dynamic topologies and varying densities. All the previously mentioned storage solutions are part of a wider category of storage designs called \ac{CRDT}, which aim to provide fast data availability and eventual data consistency.

\section{Routing}
To transmit data efficiently between two devices which are not directly connected, a routing strategy is necessary. That type of situation occurs in swarm robotics applications, where robots are often spread over a terrain and do not form a fully connected topology. Finding the best path between two points can be done with Dijkstra's algorithm \cite{dijkstra1959note}. However, such a method requires knowing the network topology in advance, and is quite computationally expensive, in the order of $\Theta((|V|+|E|)\log|V|)$ if $V$ vertices and $E$ edges are involved in the network. It is a prohibitive cost, especially if it needs to be computed periodically in dynamic topologies. Similarly, voting-based routing algorithms cannot be considered, as the voting process might introduce significant delays.

For these reasons, it is worth considering algorithms designed specifically for swarm robotics applications. By far one of the most prevalent approaches is to route data following a gradient based on one metric \cite{draves2004comparison,watteyne2009implementation}. The most used metric, perhaps because of its simplicity and efficiency, is the hop count between a source node and a target node. It is used notably in \cite{watteyne2009implementation,kuruvila2005hop,zhang2014efficient,al2019efficient}. Designs based on classic swarm intelligence algorithms can be used to find optimal routes, such as \cite{li2011slime,jiang2018toward,jiang2018effective,liao2008data,tolstaya2021learning}, but are more effective in static topologies, which makes their applications somewhat limited. Methods inspired by epidemics were shown to be efficient for disseminating data \cite{ganesan2002empirical,hui2004dynamic}. For data aggregation purposes, which is an application similar to the percolation \ac{RASS} seeks to achieve, the techniques in  \cite{jiang2018effective,liao2008data,dhand2016data} were all shown to provide good results and could thus serve as inspiration.

\section{Risk Assessment in Robot Swarms}
Of particular importance to the research objective of minimizing failures in the designed systems is the consideration of risk, which is directly correlated to breakdowns. Strategies on how to assess the presence of danger and on how to share this information among the members of the swarm are therefore necessary for \ac{RASS} and for \ac{DORA}.
The definitions and categorizations of risk suggested in \cite{xiao2020robot} are good starting points for creating risk assertions. However, they were created in the context of a formal path planner which is too resource-demanding in this current context. Also, it assumes prior knowledge of the environment's state, meaning that risk is not discovered by the robots themselves, but rather by a central operator. 


\section{Exploration Strategies}
The first possible strategy for terrain coverage is path planning. The planners suggested by \cite{undurti2010online,thiebaux2016rao,xiao2020robot} are based on a \ac{MDP} and are risk-oriented. Their shortcoming in relation to this work's objective are twofold. First, they assume at least a partial knowledge of the global state of the environment is available. This is not a valid assumption when exploring unknown environments. Second, they are only applied to single-robot systems.

Addressing the last point, several terrain coverage maximization strategies which are distributed in nature have been proposed. Many distributed exploration strategies that maximize the amount of covered terrain have been proposed. Voronoi-based coverage control techniques~\cite{luo2019voronoi,santos2019decentralized} achieve interesting results, but are more useful when prior knowledge of the environment exists. The same issue applies to methods using time-varying domains (i.e. dynamic environments). \cite{santos2019decentralized,xu2019multi}. A perhaps simpler form of exploration, and consequently more suited to \ac{DORA}'s requirements, is \ac{FBE} \cite{yamauchi1998frontier}. Several \ac{FBE} refinements have been developed: Particle Swarm Optimization \cite{wang2011frontier} and Wavefront Frontier Detector \cite{topiwala2018frontier} are two of them. However, none of \ac{FBE}-based strategies take risk into account. Because of their good performance in terms of terrain coverage, they could be used as benchmark solutions.

The system which is closest to \ac{DORA}'s objectives is the multi-robot control algorithm presented in \cite{dames2012decentralized,schwagerMultirobotControlPolicy2017}, because it maximizes the information gain during exploration in the presence of unknown hazards. Yet, it is not perfectly aligned with all non-functional requirements because it seeks an optimal exploration solution, which entails a very high computational complexity. Using it would require introducing several approximations to lower the computation load of the robots.


\section{Swarm Programming and Simulation}
Swarm programming focuses on four key properties: decentralized control, absence of leaders, absence of predefined roles, and reliance on simple and local interactions. In this sense, it differs greatly from conventional centralized cloud-based applications or even from centralized sensor arrays which are common in \ac{IoT} applications. Some programming paradigms for robot swarms, such as robot oriented like ROS, \cite{quigley2009ros}, aggregate programming like Protelis \cite{pianini2015protelis} or task-oriented programming such as Voltron \cite{mottola2014team} all have their shortcomings to address the issue of coordinating swarms of robots. Buzz \cite{pinciroliBuzz2016}, another programming language, addresses these issues and aims to meet the key properties mentioned earlier, therefore enabling an easier approach to swarm programming. Among other things, Buzz uses a custom virtual machine to run code in isolation of the operating system. It is designed to be an extensible language with high-level primitives facilitating robot interactions. A recently developed language, Koord \cite{ghosh2020koord}, could also inspire software testing mechanisms, as it is built around a strong formal method philosophy. This means it allows swarms of robots to be programmed with verifiability in mind, that is each part of the distributed algorithms should be tested in a modular way with fixed guarantees in place.

Because validating various aspects of the proposed systems requires simulations, it is necessary to use a simulator well suited to scenarios in which multiple agents are present. It is also necessary to mitigate the previously mentioned robot-simulation gap. In this regard, ARGoS  \cite{Pinciroli:SI2012} is particularly well suited, because it has the capacity to run agents on separate threads, therefore improving simulation execution speed. This has the effect of making large scale simulations accessible. Moreover, because ARGoS is a physics-based simulator, it allows to take into account the effects physical interactions between robots, their environment and themselves. Furthermore, ARGoS integrates particularly well with Buzz.

\section{Grid Mapping}
In the task of map building which is required for (or is the purpose of) robotic exploration, belief maps offer a simple and powerful tool. They show significant improvements for exploration performance over occupancy maps  \cite{stachnissMappingExplorationMobile2003}. Their use in robotic exploration is far from new maps date back as far as twenty years \cite{kobayashiSharingExploringInformation2002,kobayashiDeterminationExplorationTarget2003}.Their disadvantages are that they are limited to fixed grid sizes and are tested only with two robots, which make their application limited. Recently, in \cite{indelmanCooperativeMultirobotBelief2018}, improvements for multi-robot grid mapping have been suggested. In that article, the robots consider both the current and expected beliefs to collaborate. However, there seems to be no usage of them for storing risk beliefs. It should be noted that these maps are well suited to be stored in \ac{CRDT}s, because the cell locations can be used as keys while the beliefs are stored as values.