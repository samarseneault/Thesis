% Abstract
%
% Résumé de la recherche écrit en anglais sans être
% une traduction mot à mot du résumé écrit en français.

\chapter*{ABSTRACT}\thispagestyle{headings}
\addcontentsline{toc}{compteur}{ABSTRACT}
%
\begin{otherlanguage}{english}
Risk is omnipresent in our lives; it is quite simply impossible to ignore it. We may either avoid or confront it. Depending on the situation, both approaches have their merit. On one hand, by always avoiding danger, one will accomplish very little. On the other hand, constantly defying it may lead to one's ruin. Humans choose the best course of action based on the gravity of the situation and on their personal risk tolerance. Yet risk is not only present in our everyday lives, but also in scientific application domains. This is the case of swarm robotics, in which groups of robots often accomplish missions in dangerous environments. So how would it be possible to instill a risk management policy to these robots to make them more resilient?

The strategy adopted in the thesis in order to answer the previous question is to create a distributed storage policy which takes risk into account, as well as to create a collaborative solution to explore an unknown environment without needlessly exposing robots to danger. This way, the relevance of risk awareness is verified in two different but related swarm robotics contexts, adding diversity and value to this work's contribution. The first system described in this thesis is \textit{\acl{RASS}}, which is based on robots' potential to safely store data. It also relies on gradient-based routing to achieve data percolation towards a central point. The second idea explored in this thesis is \textit{\acl{DORA}}, which optimizes local information and risk gradients to explore an unknown environment.

Two evaluation phases are required for each developed system. First, simulations are performed to verify the conceptual sanity of the designs as well as their scaling capacity. Second, the systems are tested on physical robots to ensure they have real-world applicability. For both \ac{RASS} and \ac{DORA}, results show that just like with human behavior, reasonably factoring risk in decision processes often yields optimal results.

\ac{RASS} and \ac{DORA} offer ideas to make swarm robotics applications safer and more resilient. Therefore, by adapting and possibly improving them where needed, they could be used in diverse and complex situations which were not studied in this work, such as search and rescue scenarios or cavern exploration.

\end{otherlanguage}
