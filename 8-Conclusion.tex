\Chapter{CONCLUSION}\label{sec:Conclusion}
The objective of the research conducted for this thesis was to design risk-aware algorithms to improve the resiliency of swarm robotics systems. RASS and DORA are both meant to be used as part of other other systems, and thus to be stepping stones towards trustworthy industrial applications.

%%
%%  SYNTHESE DES TRAVAUX / SUMMARY OF WORKS
%%
\section{Summary of Works}
Texte / Text.

%%
%%  LIMITATIONS
%%
\section{Limitations}\label{sec:Limitations}
Robot-simulation gap would require further experiments.

Physical scalability, in both RASS and DORA, has not been extensively tested. This is due to a limited number of robots available for each experiment. In RASS's case, we could only use 5 CogniFlies \cite{de2021flexible}, and for DORA, we only had access to 5 KheperaIV robots.

%%
%%  AMELIORATIONS FUTURES *-/ FUTURE RESEARCH
%%
\section{Future Research}
Adapting solutions to heterogeneous swarms with different resistance to risk could prove an interesting challenge and would show the adaptability of both RASS and DORA to more diverse scenarios in which swarms can retain their group performance advantage derived from heterogeneity \cite{ferrante2015evolution}. Testing in more challenging environments (rugged terrain, terrains with hidden lines of sights, dynamic risk conditions, etc.). Further improvements could be integrated to the system, such as improvements to security by using techniques like sinkhole detection proposed by \cite{abdullah2015detecting}.